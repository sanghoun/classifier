\documentclass[11pt]{article}
\usepackage{acl2015}
\usepackage{times}
\usepackage{url}
\usepackage{latexsym}

% You can expand the titlebox if you need extra space
% to show all the authors. Please do not make the titlebox
% smaller than 5cm (the original size); we will check this
% in the camera-ready version and ask you to change it back.


\title{Generating Classifier Constructions}

\author{First Author \\
  Affiliation / Address line 1 \\
  Affiliation / Address line 2 \\
  Affiliation / Address line 3 \\
  {\tt email@domain} \\\And
  Second Author \\
  Affiliation / Address line 1 \\
  Affiliation / Address line 2 \\
  Affiliation / Address line 3 \\
  {\tt email@domain} \\}

\date{}

\begin{document}
\maketitle
\begin{abstract}
  This document contains the instructions for preparing a camera-ready
  manuscript for the proceedings of ACL-2014. The document itself
  conforms to its own specifications, and is therefore an example of
  what your manuscript should look like. These instructions should be
  used for both papers submitted for review and for final versions of
  accepted papers.  Authors are asked to conform to all the directions
  reported in this document.
\end{abstract}

\section{Introduction}

The following instructions are directed to authors of papers submitted

\section{General Instructions}

Manuscripts must be in two-column format.  Exceptions to the
\cite{bond:etal:11}

\newcite{bond:etal:11}


%% \section*{Acknowledgments}

%% The acknowledgments should go immediately before the references.  Do
%% not number the acknowledgments section. Do not include this section
%% when submitting your paper for review.

\bibliographystyle{acl}
\bibliography{classifier}

\end{document}
